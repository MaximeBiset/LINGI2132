\documentclass[10pt,a4paper]{article}
\usepackage[utf8]{inputenc}
\usepackage[english]{babel}
\usepackage{amsmath}
\usepackage{amsfonts}
\usepackage{amssymb}
\usepackage{listing}
\author{Julien De Coster Xavier Crochet}
\title{LINGI2132 Assignement 2 Report}
\begin{document}
\maketitle
\section{Multi Line Comments}
\subsection{Handwritten scanner}
Add, in the '/' case, that if another '*' is scanned, we have to skip all the characters until we scan '*/'. 
\subsection{JavaCC generated scanner}
Add in the j--.jj file a grammar rule that switch in the state \textit{multiline} if '/*' is encountered. We move from this state only if '/*' is scanned. 
\section{Conditional Expression}
\subsection{Implementation}
We add the \textit{QM '?'} token and the \textit{COLON  ':'} one, create a class JConditionalExpression and only implements the constructor and the writeToStdOut method as asked in the \textit{Énoncé}. To do so, we create the method \textit{conditionalExpression}. This method invokes the lower level, \textit{conditionalAndExpression} and if it scans a QM token, create an \textit{assignmentExpression}, check whether there is a COLON token, \textit{recursively} call itself and return a new JConditionalStatement with the previous created parameters.\\
A JConditionalStatement is a Statement with the following parameters:
\begin{itemize}
\item{JExpression \textit{condition} : the condition to test}
\item{JExpression \textit{then part} : the expression to branch to if \textit{condition} is true}
\item{JExpression \textit{else part} : the expression to branch to if \textit{condition} is false (this expression can be another JConditionalExpression)}
\end{itemize}
\subsection{Difference between the handwritten scanner an the other one}
\subsection{Tests}
We test tricky case as follow :
\begin{itemize}
\item{Comparison in the condition}
\item{Boolean in the condition}
\item{Conditional expression with conditional expression in the else part}
\item{...}
\end{itemize}
\section{For Statement}
\subsection{Implementation}
Here we have to support two type of for loop ;
\begin{enumerate}
\item{The basic one : for ( [forInit] ; [expression] ; [forUpdate] ) statement}
\item{The enhanced one : for ( FormalParameter : Expression ) Statement}
\end{enumerate}

For the first one, we have to create a forInit and a forUpdate methode to parse the concerned expression. Concerning forInit, we have to handle the \textit{final} modifier and multiples variable declarations. Concerning forUpdate, we just have to handle multiple statements. Notice that we have to parse the \textit{final} modifier in the variableDeclaration function and that modifiers are already supported by the class. However, we have to add \textit{final} modifier support into the JFormalParameter class in order to parse it when encountered.

\subsection{Difference between the handwritten scanner an the other one}
In j--.jj we have to use \textit{LOOKAHEAD(PATTERN)} in order distinguish the two for loops.  
\subsection{Tests}
\begin{itemize}
\item{Final modifier in \textit{forInit}/\textit{formalParameter}}
\item{Multiple variable declaration in \textit{forInit}}
\item{Multiple statement in \textit{ForUpdate}}
\item{Empty \textit{basicForLoop} (for(;;;))}
\end{itemize}

\section{Conclusion}
\end{document}