\documentclass[10pt,a4paper]{article}
\usepackage[utf8]{inputenc}
\usepackage[francais]{babel}
\usepackage[T1]{fontenc}
\usepackage{amsmath}
\usepackage{amsfonts}
\usepackage{amssymb}
\author{Xavier Crochet, Julien De Coster}
\title{INGI2132 \\ Assignement 1 Report}
\begin{document}
\maketitle

\section{Introduction}

We implemented first the division, the unary plus and the modulo. We wrote likewise the program \emph{primes.java} with the Eratosthenes algorithm. We used jUnit tests to test both the operators and the j-- program.

\section{Development}

The main choice we did was on the data structure of the \emph{primes.java} program. We analyzed two possibilies :

\begin{enumerate}
\item A structure with two lists. The first list contains integers from 1 to \emph{n}. The second list contains the marks. Both lists are equals in size in order to match each element from the first list to a mark from the second one. An element is marked as 0 if it is potentially a prime number or as 1 if it is identified as a multiple of a previous number of the list. We use the term \emph{potentially} because all the elements of the mark list are initialized at 0 except 0 and 1, initialized at 1.

\item A structure with an unique linked list. Here the strategy is to remove every element identified as a multiple of a previous number. The benefit of this method is clear : we don't have to visit elements wich are not relevant because already identied.
\end{enumerate}

We choose the first strategy mainly for performance. Accessing elements and remove them from a linked list is costly. Our mesures showed that the first strategy took more or less ten times less time with wide values (e.g. 50 000).

\section{Testing}

We tested both the division, plus and modulo with priorities and the j-- program.

For the operators we used first basic tests (e.g. "4 / 2" or "4 mod 5") and then we tested the priorities (e.g. "3 + 4 / 5").

We used another strategy for the j-- program. We mainly tested the extreme values : "What happen if we put negative numbers ?". We tested for values like 0, 1 (intialized at 1) and 2.

\end{document}